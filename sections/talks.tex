\begin{rSection}{Other Presentations}

%Thesis 
\begin{rSubsection}{Synopsis 2021}{April 2021}{McMaster University}{$k$-special $3$-smooth Representations and the Collatz Conjecture}
    \begin{itemize}
      \addtolength\itemsep{-0.5em}
      \item A 15-minute expository talk on a formulation of the Collatz conjecture by a family of Diophantine equations and a conjecturally sparse set of numbers that are `almost' solutions.
    \end{itemize}
\end{rSubsection}
\smallskip

%%CANDEV 
%\begin{rSubsection}{CANDEV}{January 2020}{Government of Canada}{Transformer Embeddings to  Identify Course Redundancies}
%    \begin{itemize}
%      \addtolength\itemsep{-0.5em}
%      \item Gave a short talk on our use of transfer-learning with a transformer model to cluster courses offered by the Canadian School of Public Service and identify redundancies in course offerings.
%    \end{itemize}
%\end{rSubsection}
%\smallskip

%Prime Numbers 
%\begin{rSubsection}{Synopsis 2019}{April 2019}{McMaster University}{Prime Distribution by Linear Flow on the Torus}
%	%\smallskip
%	A 15-minute expository talk on the primary findings of a 4-month project investigating prime distributions over non-intersecting curves on closed surfaces.
%\end{rSubsection}
%\smallskip

%\vspace{4.5em}


%{\bf Predicting Drug-Drug Interactions in the Body using Minimal-Input Neural Networks}\\
%This was a data science competition focusing on the effects of recreational drugs. We designed a neural network that would predict, using only experimental properties of a compound, and with no knowledge of drug structure, whether or not two compounds would interact in the body. This achieved a state-of-the-art accuracy of 94.2\%. We presented a seminar on this paper at York University, and the paper has been published in the STEM Fellowship Journal. \\

%{\bf Using Agent-Based Modelling to Simulate Tumour Growth and Progression}\\
%Using agent-based modelling, canine transmissible venereal tumours were simulated, and the effects of various treatment methods on this growth were examined. Specifically, we investigated the immunohistological environment of the tumour and how changing MHC expression and various Ig concentrations affected tumour spread and virility. \\

% {\bf An Experiment in Plant-Animal Interactions}\\
% In this project, we designed and carried out a factorial experiment to assess the type and extent of herbivory by \textit{Myzus persicae} on \textit{Arabadopsis thaliana}. Specifically, we examined the effect that soil-nitrate content had on this relationship. A paper was written discussing the findings and statistical analyses applied, after which a small presentation was given to summarize the results. \\

\end{rSection}